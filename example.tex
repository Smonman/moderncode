\documentclass[10pt]{article}
\usepackage{moderncode}

\begin{document}
\section{Moderncode}

\begin{moderncode}[C][adjusted title={Title}]
int main(int ac, char *av[])
{
	printf("Hello, World");
	return 0;
}
\end{moderncode}

\begin{moderncode}[C][adjusted title={This title is very very very very very very very very very long}]
int main(int ac, char *av[])
{
	printf("Hello, World");
	return 0;
}
\end{moderncode}

\begin{moderncode}[C]
int main(int ac, char *av[])
{
	printf("Hello, World");
	return 0;
}
\end{moderncode}

\begin{moderncode}[C][adjusted title={This is a very long code}]
int main(int ac, char *av[])
{
	printf("Hello, World");
	printf("Hello, World");
	printf("Hello, World");
	printf("Hello, World");
	printf("Hello, World");
	printf("Hello, World");
	printf("Hello, World");
	printf("Hello, World");
	printf("Hello, World");
	printf("Hello, World");
	printf("Hello, World");
	printf("Hello, World");
	printf("Hello, World");
	printf("Hello, World");
	printf("Hello, World");
	printf("Hello, World");
	return 0;
}
\end{moderncode}

\subsection{Output}

\begin{moderncodeout}
Enter a positive integer: 100
Fibonacci Series: 0, 1, 1, 2, 3, 5, 8, 13, 21, 34, 55, 89
\end{moderncodeout}

\subsection{Inline}

\subsubsection{Inline Code}

This is an inline modern code display: \moderncodeinline{\LaTeX}. It is also possible to define a language: \moderncodeinline[SQL]{SELECT pg_relation_size('title_basics');}.

\subsection{Inline Key}

It also supports a key-like-style inline element: \moderncodekey{Ctrl} + \moderncodekey{C}

\section{Lstlisting}

\begin{lstlisting}[caption=Example in C++,language=c++]
#include <iostream>
using namespace std;

int main() {
	int n, t1 = 0, t2 = 1, nextTerm = 0;

	cout << "Enter the number of terms: ";
	cin >> n;

	cout << "Fibonacci Series: ";

	for (int i = 1; i <= n; ++i) {
		// prints the first two terms.
		if(i == 1) {
			cout << t1 << ", ";
			continue;
		}
		if(i == 2) {
			cout << t2 << ", ";
			continue;
		}
		nextTerm = t1 + t2;
		t1 = t2;
		t2 = nextTerm;

		cout << nextTerm << ", ";
	}
	return 0;
}
\end{lstlisting}

\subsection{Output}

\begin{lstlisting}[style=lstoutput]
Enter a positive integer: 100
Fibonacci Series: 0, 1, 1, 2, 3, 5, 8, 13, 21, 34, 55, 89
\end{lstlisting}

\subsection{Inline}

\subsubsection{Inline Code}

\lstinline{\LaTeX}

\end{document}